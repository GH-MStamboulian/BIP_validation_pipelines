\section{Call Concordance Tables}
\label{a:call_concordance_tables_raw}

 \captionsetup{width=.65\textwidth}
 \begin{table}[H]
 \centering
 \caption{\textbf{Call comparison types.}}
  \label{t:collection12_concordance_types}
 \begin{tabular}{|c|c|c|}
      \hline
       & \textbf{G360 CDx+} & \textbf{G360 CDx-} \\ \hline
      \textbf{LBP70+} & $a$ & $b$ \\ \hline
      \textbf{LBP70-} & $c$ & $d$ \\ \hline
 \end{tabular}
 \caption*{The variables
 in each cell of the table represent the number of calls. For example,
 $b$ is the number of calls negative by G360 CDx and positive by LBP70.}
 \end{table}
 
\begin{enumerate}
        \item PPA and NPA for each sample collection and variant category
		 
		 \captionsetup{justification=raggedright, singlelinecheck=off,skip=0pt}
		 \begin{table}[H]
		 \centering
		 \begin{threeparttable}
		 \caption{\textbf{PPA and NPA for each sample collection and variant category.}}
		 \label{t:coll123_variant_category_raw_binomial_2test}
		 \begin{tabular}{|l|L{0.7in}|l|l|l|l|l|l|l|}
\hline
\rowcolor[gray]{.85}\textbf{SC} &  {}  &  {}  & \textbf{$a$} & \textbf{$b$} & \textbf{$c$} & \textbf{$d$} & \textbf{PPA} & \textbf{                         NPA }\\ \hline
\multirow{6}{*}{1} & \multirow{4}{0.7in}{Clinically Relevant} & SNV &   34 &    3 &    3 &    7960 &    91.9 (78.1 - 98.3) &    99.962 (99.890 - 99.992) \\ \cline{3-9}
  &            & Indel &    6 &    0 &    0 &    9794 &  100.0 (54.1 - 100.0) &  100.000 (99.962 - 100.000) \\ \cline{3-9}
  &            & CNA &    7 &    3 &    1 &     189 &    70.0 (34.8 - 93.3) &    99.474 (97.103 - 99.987) \\ \cline{3-9}
  &            & Fusion &    0 &    0 &    0 &     400 &                    NA &  100.000 (99.082 - 100.000) \\ \cline{2-9}
  & \multirow{2}{0.7in}{Panel-wide} & SNV &  286 &   47 &   50 &  399017 &    85.9 (81.7 - 89.4) &    99.987 (99.983 - 99.991) \\ \cline{3-9}
  &            & Indel &   36 &    6 &    6 &   75152 &    85.7 (71.5 - 94.6) &    99.992 (99.983 - 99.997) \\ \hline
\multirow{6}{*}{2} & \multirow{4}{0.7in}{Clinically Relevant} & SNV &   76 &    4 &    4 &    7916 &    95.0 (87.7 - 98.6) &    99.949 (99.871 - 99.986) \\ \cline{3-9}
  &            & Indel &   31 &    2 &    1 &    9766 &    93.9 (79.8 - 99.3) &   99.990 (99.943 - 100.000) \\ \cline{3-9}
  &            & CNA &   23 &    3 &    1 &     173 &    88.5 (69.8 - 97.6) &    99.425 (96.840 - 99.985) \\ \cline{3-9}
  &            & Fusion &    3 &    0 &    0 &     397 &  100.0 (29.2 - 100.0) &  100.000 (99.075 - 100.000) \\ \cline{2-9}
  & \multirow{2}{0.7in}{Panel-wide} & SNV &  320 &   87 &   51 &  398942 &    78.6 (74.3 - 82.5) &    99.987 (99.983 - 99.990) \\ \cline{3-9}
  &            & Indel &   53 &    8 &    4 &   75135 &    86.9 (75.8 - 94.2) &    99.995 (99.986 - 99.999) \\ \hline
\multirow{6}{*}{3} & \multirow{4}{0.7in}{Clinically Relevant} & SNV &   82 &    7 &   10 &   12381 &    92.1 (84.5 - 96.8) &    99.919 (99.852 - 99.961) \\ \cline{3-9}
  &            & Indel &   52 &   13 &   15 &   15208 &    80.0 (68.2 - 88.9) &    99.901 (99.838 - 99.945) \\ \cline{3-9}
  &            & CNA &   17 &    6 &    1 &     288 &    73.9 (51.6 - 89.8) &    99.654 (98.087 - 99.991) \\ \cline{3-9}
  &            & Fusion &   34 &    1 &    4 &     585 &    97.1 (85.1 - 99.9) &    99.321 (98.270 - 99.815) \\ \cline{2-9}
  & \multirow{2}{0.7in}{Panel-wide} & SNV &  596 &  139 &  136 &  622193 &    81.1 (78.1 - 83.9) &    99.978 (99.974 - 99.982) \\ \cline{3-9}
  &            & Indel &  150 &   42 &   36 &  117084 &    78.1 (71.6 - 83.8) &    99.969 (99.957 - 99.978) \\ \hline
\end{tabular}
		 \caption*{SC indicates the sample collection. $a$, $b$, $c$, and $d$ are defined 
		  in \cref{t:collection12_concordance_types}. The 95\% Clopper-Pearson intervals are
		  shown in parantheses with the PPA and NPA values.}
		 \end{threeparttable}
		 \end{table}

	\item PPA and NPA for each sample collection and variant class
	
		 \begin{table}[H]
		 \centering
		 \begin{threeparttable}
		 \caption{\textbf{Sample collection 1, call concordance by variant class.}}
		 \label{t:coll1_variant_classspecial_raw_binomial_2test}
		 \begin{tabular}{|l|l|l|l|l|l|l|}
\hline
\rowcolor[gray]{.85}{}                           & \textbf{$a$} & \textbf{$b$} & \textbf{$c$} & \textbf{$d$} & \textbf{PPA} & \textbf{                         NPA }\\ \hline
BRAF Activating SNV         &    4 &    0 &    0 &     96 &  100.0 (39.8 - 100.0) &  100.000 (96.230 - 100.000) \\ \hline
BRCA1 Inactivating SNV      &    0 &    0 &    1 &   2899 &                    NA &    99.966 (99.808 - 99.999) \\ \hline
BRCA2 Inactivating SNV      &    0 &    1 &    0 &   1799 &      0.0 (0.0 - 97.5) &  100.000 (99.795 - 100.000) \\ \hline
EGFR L858R                  &    4 &    0 &    0 &     96 &  100.0 (39.8 - 100.0) &  100.000 (96.230 - 100.000) \\ \hline
EGFR T790M                  &    0 &    0 &    1 &     99 &                    NA &    99.000 (94.554 - 99.975) \\ \hline
KRAS Activating SNVs        &   23 &    1 &    0 &   1276 &    95.8 (78.9 - 99.9) &  100.000 (99.711 - 100.000) \\ \hline
NRAS Activating SNVs        &    3 &    1 &    1 &   1095 &    75.0 (19.4 - 99.4) &    99.909 (99.493 - 99.998) \\ \hline
Other EGFR activating SNVs  &    0 &    0 &    0 &    600 &                    NA &  100.000 (99.387 - 100.000) \\ \hline
BRCA2 Inactivating Indel    &    0 &    0 &    0 &   5600 &                    NA &  100.000 (99.934 - 100.000) \\ \hline
BRCA1 Inactivating Indel    &    0 &    0 &    0 &   2200 &                    NA &  100.000 (99.832 - 100.000) \\ \hline
EGFR Activating Indel       &    5 &    0 &    0 &    795 &  100.0 (47.8 - 100.0) &  100.000 (99.537 - 100.000) \\ \hline
ERBB2 Activating Indel      &    1 &    0 &    0 &    299 &   100.0 (2.5 - 100.0) &  100.000 (98.774 - 100.000) \\ \hline
Other EGFR activating Indel &    0 &    0 &    0 &    900 &                    NA &  100.000 (99.591 - 100.000) \\ \hline
Homopolymer                 &    4 &    1 &    1 &  24194 &    80.0 (28.4 - 99.5) &   99.996 (99.977 - 100.000) \\ \hline
Long Indel                  &    0 &    0 &    0 &   2500 &                    NA &  100.000 (99.853 - 100.000) \\ \hline
ERBB2 Amplification         &    5 &    0 &    0 &     95 &  100.0 (47.8 - 100.0) &  100.000 (96.191 - 100.000) \\ \hline
MET Amplification           &    2 &    3 &    1 &     94 &     40.0 (5.3 - 85.3) &    98.947 (94.274 - 99.973) \\ \hline
NTRK1 Fusion                &    0 &    0 &    0 &    100 &                    NA &  100.000 (96.378 - 100.000) \\ \hline
RET Fusion                  &    0 &    0 &    0 &    400 &                    NA &  100.000 (99.082 - 100.000) \\ \hline
ROS1 Fusion                 &    0 &    0 &    0 &    400 &                    NA &  100.000 (99.082 - 100.000) \\ \hline
ALK Fusion                  &    0 &    0 &    0 &    100 &                    NA &  100.000 (96.378 - 100.000) \\ \hline
\end{tabular}
		 \caption*{SC 
		 indicates the sample collection. $a$, $b$, $c$, and $d$ are 
		 defined in \cref{t:collection12_concordance_types}. 
		 The 95\% Clopper-Pearson inteveras are
		  shown in parantheses with the PPA and NPA values.}
		 \end{threeparttable}
		 \end{table}
		 
		 \begin{table}[H]
		 \centering
		 \begin{threeparttable}
		 \caption{\textbf{Sample collection 2, call concordance by variant class.}}
		 \label{t:coll2_variant_classspecial_raw_binomial_2test}
		 \begin{tabular}{|l|l|l|l|l|l|l|}
\hline
\rowcolor[gray]{.85}{}                           & \textbf{$a$} & \textbf{$b$} & \textbf{$c$} & \textbf{$d$} & \textbf{PPA} & \textbf{                         NPA }\\ \hline
BRAF Activating SNV         &   12 &    0 &    0 &     88 &  100.0 (73.5 - 100.0) &  100.000 (95.895 - 100.000) \\ \hline
BRCA1 Inactivating SNV      &    1 &    0 &    0 &   2899 &   100.0 (2.5 - 100.0) &  100.000 (99.873 - 100.000) \\ \hline
BRCA2 Inactivating SNV      &    3 &    0 &    0 &   1797 &  100.0 (29.2 - 100.0) &  100.000 (99.795 - 100.000) \\ \hline
EGFR L858R                  &   10 &    0 &    0 &     90 &  100.0 (69.2 - 100.0) &  100.000 (95.984 - 100.000) \\ \hline
EGFR T790M                  &   10 &    0 &    0 &     90 &  100.0 (69.2 - 100.0) &  100.000 (95.984 - 100.000) \\ \hline
KRAS Activating SNVs        &   15 &    3 &    2 &   1280 &    83.3 (58.6 - 96.4) &    99.844 (99.438 - 99.981) \\ \hline
NRAS Activating SNVs        &   20 &    0 &    2 &   1078 &  100.0 (83.2 - 100.0) &    99.815 (99.333 - 99.978) \\ \hline
Other EGFR activating SNVs  &    5 &    1 &    0 &    594 &    83.3 (35.9 - 99.6) &  100.000 (99.381 - 100.000) \\ \hline
BRCA2 Inactivating Indel    &    5 &    0 &    1 &   5594 &  100.0 (47.8 - 100.0) &   99.982 (99.900 - 100.000) \\ \hline
BRCA1 Inactivating Indel    &    1 &    1 &    0 &   2198 &     50.0 (1.3 - 98.7) &  100.000 (99.832 - 100.000) \\ \hline
EGFR Activating Indel       &   14 &    1 &    0 &    785 &    93.3 (68.1 - 99.8) &  100.000 (99.531 - 100.000) \\ \hline
ERBB2 Activating Indel      &    6 &    0 &    0 &    294 &  100.0 (54.1 - 100.0) &  100.000 (98.753 - 100.000) \\ \hline
Other EGFR activating Indel &    5 &    0 &    0 &    895 &  100.0 (47.8 - 100.0) &  100.000 (99.589 - 100.000) \\ \hline
Homopolymer                 &    5 &    0 &    0 &  24195 &  100.0 (47.8 - 100.0) &  100.000 (99.985 - 100.000) \\ \hline
Long Indel                  &    0 &    1 &    0 &   2499 &      0.0 (0.0 - 97.5) &  100.000 (99.852 - 100.000) \\ \hline
ERBB2 Amplification         &   12 &    1 &    0 &     87 &    92.3 (64.0 - 99.8) &  100.000 (95.849 - 100.000) \\ \hline
MET Amplification           &   11 &    2 &    1 &     86 &    84.6 (54.6 - 98.1) &    98.851 (93.762 - 99.971) \\ \hline
NTRK1 Fusion                &    0 &    0 &    0 &    100 &                    NA &  100.000 (96.378 - 100.000) \\ \hline
RET Fusion                  &    0 &    0 &    0 &    400 &                    NA &  100.000 (99.082 - 100.000) \\ \hline
ROS1 Fusion                 &    0 &    0 &    0 &    400 &                    NA &  100.000 (99.082 - 100.000) \\ \hline
ALK Fusion                  &    3 &    0 &    0 &     97 &  100.0 (29.2 - 100.0) &  100.000 (96.268 - 100.000) \\ \hline
\end{tabular}
		 \end{threeparttable}
		 \end{table}
		 
		 \begin{table}[H]
		 \centering
		 \begin{threeparttable}
		 \caption{\textbf{Sample collection 3, call concordance by variant class.}}
		 \label{t:coll3_variant_classspecial_raw_binomial_2test}
		 \begin{tabular}{|l|l|l|l|l|l|l|}
\hline
\rowcolor[gray]{.85}{}                           & \textbf{$a$} & \textbf{$b$} & \textbf{$c$} & \textbf{$d$} & \textbf{PPA} & \textbf{                         NPA }\\ \hline
BRAF Activating SNV         &    5 &    0 &    1 &    150 &  100.0 (47.8 - 100.0) &    99.338 (96.366 - 99.983) \\ \hline
BRCA1 Inactivating SNV      &   14 &    2 &    4 &   4504 &    87.5 (61.7 - 98.4) &    99.911 (99.773 - 99.976) \\ \hline
BRCA2 Inactivating SNV      &    9 &    0 &    0 &   2799 &  100.0 (66.4 - 100.0) &  100.000 (99.868 - 100.000) \\ \hline
EGFR L858R                  &    4 &    0 &    1 &    151 &  100.0 (39.8 - 100.0) &    99.342 (96.389 - 99.983) \\ \hline
EGFR T790M                  &    9 &    1 &    2 &    144 &    90.0 (55.5 - 99.7) &    98.630 (95.139 - 99.834) \\ \hline
KRAS Activating SNVs        &   22 &    2 &    2 &   2002 &    91.7 (73.0 - 99.0) &    99.900 (99.640 - 99.988) \\ \hline
NRAS Activating SNVs        &    5 &    2 &    0 &   1709 &    71.4 (29.0 - 96.3) &  100.000 (99.784 - 100.000) \\ \hline
Other EGFR activating SNVs  &   14 &    0 &    0 &    922 &  100.0 (76.8 - 100.0) &  100.000 (99.601 - 100.000) \\ \hline
BRCA2 Inactivating Indel    &   22 &    9 &   10 &   8695 &    71.0 (52.0 - 85.8) &    99.885 (99.789 - 99.945) \\ \hline
BRCA1 Inactivating Indel    &   12 &    4 &    4 &   3412 &    75.0 (47.6 - 92.7) &    99.883 (99.700 - 99.968) \\ \hline
EGFR Activating Indel       &   12 &    0 &    1 &   1235 &  100.0 (73.5 - 100.0) &    99.919 (99.550 - 99.998) \\ \hline
ERBB2 Activating Indel      &    4 &    0 &    0 &    464 &  100.0 (39.8 - 100.0) &  100.000 (99.208 - 100.000) \\ \hline
Other EGFR activating Indel &    2 &    0 &    0 &   1402 &  100.0 (15.8 - 100.0) &  100.000 (99.737 - 100.000) \\ \hline
Homopolymer                 &   45 &   12 &    6 &  37689 &    78.9 (66.1 - 88.6) &    99.984 (99.965 - 99.994) \\ \hline
Long Indel                  &   10 &    3 &    4 &   3883 &    76.9 (46.2 - 95.0) &    99.897 (99.737 - 99.972) \\ \hline
ERBB2 Amplification         &    3 &    1 &    0 &    152 &    75.0 (19.4 - 99.4) &  100.000 (97.602 - 100.000) \\ \hline
MET Amplification           &   14 &    5 &    1 &    136 &    73.7 (48.8 - 90.9) &    99.270 (96.000 - 99.982) \\ \hline
NTRK1 Fusion                &    5 &    0 &    0 &    151 &  100.0 (47.8 - 100.0) &  100.000 (97.587 - 100.000) \\ \hline
RET Fusion                  &   11 &    1 &    2 &    610 &    91.7 (61.5 - 99.8) &    99.673 (98.825 - 99.960) \\ \hline
ROS1 Fusion                 &   11 &    0 &    0 &    613 &  100.0 (71.5 - 100.0) &  100.000 (99.400 - 100.000) \\ \hline
ALK Fusion                  &    7 &    0 &    2 &    147 &  100.0 (59.0 - 100.0) &    98.658 (95.235 - 99.837) \\ \hline
\end{tabular}
		 \end{threeparttable}
		 \end{table}
		 
\end{enumerate}